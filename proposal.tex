\documentclass[a4paper, 11pt]{article}

\usepackage{kotex} % Comment this out if you are not using Hangul
\usepackage{fullpage}
\usepackage{hyperref}
\usepackage{amsthm}
\usepackage[numbers,sort&compress]{natbib}

\theoremstyle{definition}
\newtheorem{exercise}{Exercise}

\begin{document}
%%% Header starts
\noindent{\large\textbf{IS-521 Activity Proposal}\hfill
                \textbf{Ji Hyeon Yoon}} \\
         {\phantom{} \hfill \textbf{ggoboogy}} \\
         {\phantom{} \hfill Due Date: April 15, 2017} \\
%%% Header ends

\section{Activity Overview}
본 과제에서는 구글 브레인에서 공개한 머신러닝 오픈소스 라이브러리 Tensorflow~\cite{tensorflow}를 이용하여 CAPTCHA를 자동으로 판별하는 프로그램을 작성한다. 
본 과제의 주요 목적은 TensorFlow를 통해 머신러닝을 구현하는 방법을 배우고, 이를 활용하여 향후 여러가지 보안문제를 해결하는 것이다. 
본 과제는 가장 기본적인 숫자 및 문자 형태의 CAPTCHA를 판별해 내는 것을 1차 목표로 한다. 
숫자 및 문자 형태의 CAPTCHA의 경우 사용자에게 상당히 annoying한 작업일 뿐만 아니라 높은 정확도로 자동 판별이 가능하기 때문에, 
최근 구글, 드롭박스 등의 유명 사이트에서는 기존의 CAPTCHA 대신 reCAPTCHA~\cite{recaptcha}를 사용하고 있다. 
이는 IP주소, 쿠키정보, 마우스 포인터 움직임 등을 분간하여 Turing Test를 수행하는 방식으로, 
사용자는 단순히 \texttt{I'm not a robot.} 이라는 문자열 옆에 위치한 체크박스를 클릭하는 것만으로 테스트를 완료하게 된다.
현재 reCAPTCHA에서 판별에 사용하는 정확한 판단 벡터의 경우 보안상의 문제로 공개되어 있지 않은 상태이다. 
따라서 본 과제에서는 reCAPTCHA를 우회하는 경우 추가점수를 부여한다. 

\section{Exercises}

\begin{exercise}
먼저 TensorFlow의 사용 방법을 익히기 위해 TensorFlow Tutorial 중 하나인 \texttt{MNIST For ML Beginners}~\cite{tftutorial}를 수행한다. 
MNIST는 0 부터 9 사이에 해당하는 숫자 이미지로 이루어진 데이터베이스로, 
각각의 이미지는 28x28의 크기를 가지며 각각은 픽셀 밝기 값(Intensity)으로 구성되어 있다. 
본 활동에서는 Softmax Regression을 이용하여 각각의 숫자 이미지를 구분하는 코드를 작성한다. 
\end{exercise}

\begin{exercise}
두 번째 활동에서는 TensorFlow를 이용하여 실제로 사용되는 CAPTCHA 이미지를 자동 판별하는 프로그램을 작성한다. 
본 활동의 경우 CAPTCHA 데이터셋 수집에 있어 두 가지 방식으로 나눠질 수 있다. 
첫 번째는 ~\cite{captchaset}와 같이 온라인상에 공개되어있는 기존 데이터 셋을 활용하는 방법이고, 
두 번째는 일반적인 CAPTCHA Generator~\cite{generator1, generator2, generator3} 또는 
특정 사이트 (e.g., Facebook, naver) 에서 사용하는 Generator로부터 CAPTCHA 이미지 데이터셋을 직접 수집하는 것이다.
데이터 수집 또한 과제의 한 부분으로 포함될 수 있으나, 시간이 부족할 것으로 예상되기 때문에 학생들은 사전에 수집된 데이터를 부여받는 방향으로 한다.
전체 데이터셋은 모든 학생이 동일해야 하며, 트레이닝셋의 경우 전체 데이터셋 내에서 각자 자유롭게 가감이 가능하다. 
\end{exercise} 

\begin{exercise} (Optional)
본 활동에서는 근래 구글, 드롭박스 등의 대형 사이트에서 자주 사용되는 reCAPTCHA를 우회하는 프로그램을 작성한다. 
앞서 언급한 바와 같이 현재 정확한 판별 벡터가 공개되지 않은 상태이기 때문에 본 활동은 선택활동이며 우회시 추가점수가 부여된다.
\end{exercise}

\section{Expected Solutions}
\texttt{Excersice 1}의 경우 Tutorial Step을 차례대로 수행할 경우 일반적으로 약 92\%의 정확도를 나타낸다. 
여기서 약간의 수정을 통해 정확도를 높일 수 있으며, 가장 높은 정확도를 도출한 학생에게 추가점수를 부여할 수 있다. \\\\
\texttt{Excersice 2}의 경우 알고리즘, feature selection, 가중치 등 구현에 따라 정확도가 달라진다. 
최대한 높은 정확도를 도출해내는 것이 본 과제의 목표이기 때문에 학생들은 FP, FN 및 정확도에 따라 점수를 부여받을 수 있다.
정확도 산출을 위한 테스트셋의 경우 학생들에게 제공되지 않았던 새로운 데이터셋을 사용한다 (즉, 트레이닝셋 $\neq$ 테스트셋).

\bibliography{references}
\bibliographystyle{plainnat}

\end{document}
