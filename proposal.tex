\documentclass[a4paper, 11pt]{article}

\usepackage{kotex} % Comment this out if you are not using Hangul
\usepackage{fullpage}
\usepackage{hyperref}
\usepackage{amsthm}
\usepackage[numbers,sort&compress]{natbib}

\theoremstyle{definition}
\newtheorem{exercise}{Exercise}

\begin{document}
%%% Header starts
\noindent{\large\textbf{IS-521 Activity Proposal}\hfill
                \textbf{Ji Hyeon Yoon}} \\
         {\phantom{} \hfill \textbf{ggoboogy}} \\
         {\phantom{} \hfill Due Date: April 15, 2017} \\
%%% Header ends

\section{Activity Overview}
본 과제에서는 구글 브레인에서 공개한 머신러닝 오픈소스 라이브러리 Tensorflow~\cite{tensorflow}를 이용하여 CAPTCHA를 자동으로 판별하는 프로그램을 작성한다. 본 과제의 주요 목적은 TensorFlow를 통해 머신러닝을 구현하는 방법을 배우고, 이를 활용하여 향후 여러가지 보안문제를 해결하는 것이다. 본 과제는 가장 기본적인 숫자 및 문자 형태의 CAPTCHA를 판별해 내는 것을 1차 목표로 한다. 숫자 및 문자 형태의 CAPTCHA의 경우 사용자에게 상당히 annoying한 작업일 뿐만 아니라 높은 정확도로 자동 판별이 가능하기 때문에, 최근 구글, 드롭박스 등의 유명 사이트에서는 기존의 CAPTCHA 대신 reCAPTCHA~\cite{recaptcha}를 사용하고 있다. 이는 IP주소, 쿠키정보, 마우스 포인터 움직임 등을 분간하여 Turing Test를 수행하는 방식으로, 사용자는 단순히 \texttt{I'm not a robot.} 이라는 문자열 옆에 위치한 체크박스를 클릭하는 것만으로 테스트를 완료하게 된다. 현재 reCAPTCHA에서 사용하는 정확한 판단 벡터의 경우 보안상의 문제로 공개되어 있지 않은 상태이다. 따라서 본 과제에서는 reCAPTCHA를 우회하는 경우 추가점수를 부여한다. 

\section{Exercises}

Describe a series of exercises that students will carry out. (학생들이 하게
될 연습문제를 순차적으로 서술.)

\begin{exercise}

  In this exercise, you do ...

\end{exercise}

\begin{exercise}

  In this exercise, you do ...

\end{exercise}

\begin{exercise}

  In this exercise, you do ...

\end{exercise}

\section{Expected Solutions}

Put expected solutions here.
(예상되는 답안에 대해서 서술.)

\bibliography{references}
\bibliographystyle{plainnat}

\end{document}
